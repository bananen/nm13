\documentclass[a4paper]{scrreprt}

\usepackage[german]{babel}
\usepackage[utf8]{inputenc}
\usepackage[T1]{fontenc}
\usepackage{ae}
\usepackage{amssymb}
\usepackage{graphicx}
\usepackage{hyperref}

\begin{document}

\title{Numerik Zusammenfassung}
\author{Fedor Scholz}
\maketitle

\tableofcontents
\vspace{1cm}

\chapter{Lineare Gleichungssysteme}

Problem: Finde Loesungsvektor x fuer\\\\
$Ax = b$\\\\
Das Problem besitzt genau dann eine eindeutige Loesung x*, wenn A invertierbar ist.
Theoretische Loesung: $x* = A^{-1}b$

\section{Gaußsches Eliminationsverfahren}

Problem: Finde Loesungsvektor x fuer\\\\
$Ax = b$, wobei A invertierbar\\\\
Fuer jede invertierbare Matrix A existiert eine Permutationsmatrix P derart, dass eine (dazu) eindeutige Dreieckszerlegung $PA = LR$ moeglich ist, wobei R eine obere Dreickecksmatrix und L eine untere unipotente Dreiecksmatrix ist.

\subsection{Algorithmus}

\begin{itemize}
	\item Bestimme Matrizen P, L und R mit $PA = LR$
	\item Loese $Lc = Pc$ (Vorwaertssubstitution)
	\item Loese $Rx = c$ (Rueckwaertssubstitution)
\end{itemize}

Aufwand der LR-Zerlegung: $\frac{n^3}{3} - \frac{n}{3}$

\chapter{Einschub: Gleitpunktrechnung, Matrixnormen}

\section{Gleitpunktrechnung}

Darstellung einer Gleitpunktzahl: $z = a * d^e$ mit $e_{min} \leq e \leq e_{max}$

\section{Matrixnormen}

Eine Abbildung von einem Vektorraum in die reellen Zahlen heißt eine Norm, wenn gilt:
\begin{itemize}
	\item $||v|| \geq 0$ und ($||v|| = 0 \Leftrightarrow v = 0$) (positive Definitheit)
	\item $||\alpha v|| = |\alpha|||v||$ (Homogenitaet)
	\item $||v_1 + v_2|| \leq ||v_1|| + ||v_2||$ (Dreiecksungleichung)
\end{itemize}

Eine Matrixnorm passt zu einer gegebenen Vektornorm genau dann, wenn\\
$||Ax|| \leq ||A||||x||$\\\\

Sei $||.||$ eine beliebige Norm, dann definieren wir die zugehoerige Matrixnorm als\\
$||A|| := \sup\limits_{x \neq 0}\frac{||Ax||}{||x||}$\\\\

\section{Kondition linearer Gleichungssysteme}

$cond(A) := ||A||||A^{-1}||$\\\\

Eigenschaften:\\
$cond(A) = cond(\alpha A)$\\
$cond(A) = \frac{max_{||y|| = 1}||Ay||}{min_{||z|| = 1}||Az||}$

\section{Cholesky-Verfahren fuer symmetrische, positiv definite Matrizen}

Sei A symmetrisch und positiv definit, dann gilt fuer die Zerlegung $A = LR$: $R = DL^T$, wobei D eine positiv definite Diagonalmatrix ist. Da $D = diag(d_i)$ positiv definit, existiert $D^\frac{1}{2} = diag(\sqrt{d_i})$ und daher die Cholesky-Zerlegung:\\
$A = \overline{L}\overline{L}^T$\\
mit unterer Dreiecksmatrix $\overline{L} = LD^\frac{1}{2}$

\subsection{Rechenaufwand}

$\frac{1}{6}n^3$

\subsection{Algorithmus}
\begin{itemize}
	\item Bestimme mit dem Cholesky-Verfahren $\overline{L}$ mit $A = \overline{L}\overline{L}^T$ (Cholesky-Zerlegung)
	\item Loese $\overline{L}c = b$ (Vorwaertssubstitution)
	\item Loese $\overline{L}^Tx = c$ (Rueckwaertssubstitution)
\end{itemize}


\chapter{Misc}

\section{Cholesky-Zerlegung}

\begin{itemize}
	\item Bestimme mit dem Cholesky-Verfahren $\overline{L}$ mit $A = \overline{L} * \overline{L}^T$ (Cholesky-Zerlegung)
	\item Loese $\overline{L}c = b$ (Vorwaertssubstitution)
	\item Loese $\overline{L}^Tx = c$ (Rueckwaertssubstitution)
\end{itemize}

Aufwand: $\frac{1}{6}n^3$

\section{QR-Zerlegung}

\begin{itemize}
	\item Bestimme Matrizen Q und R mittels Householder-Transformationen mit $A = QR$ (QR-Zerlegung)
	\item Loese $Qc = b$ ($Q^{-1} = Q^T$, also $c = Q^Tb$)
	\item Loese $Rx = c$ (Rueckwaertssubstitution)
\end{itemize}

\end{document}
